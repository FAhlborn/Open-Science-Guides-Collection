\documentclass{article}

                
\usepackage[backend=biber,hyperref=false,citestyle=authoryear,bibstyle=authoryear]{biblatex}
                
\bibliography{bibliography}
            
\begin{document}

\title{OSG002 - Open Science and Knowledge Justice}

\maketitle


Authors: Kaan Ilgaz, Ümit Günes, My Linh Nguyen Thi, Lorenzo Vassao


Tags / topics (4): Knowledge Justice; 


\subsection{\textbf{Knowledge Justice: An Opportunity for Counter-expertise in Security vs. Science Debates}}\label{H7151279}



\emph{Knowledge justice} provides a conceptual framework to apply principles of social justice in environments of competing interests regarding science. Both knowledge and its making can be seen as a good to be distributed, including all voices for whom the science will matter. In this framework, knowledge production is shared among a broader constituency of knowers representing both local and cosmopolitan voices. The problem of knowledge injustice can be seen in the U.S. government’s recent attempt to secure scientific knowledge about H5N1 or avian bird flu virus. The censorship produced a global debate between scientists and policy-makers over how to balance the nation-state’s desire for security with the life science’s tradition of open and shared research. This conundrum, known as the dual-use dilemma, obscures larger questions that lie outside of expert-centered domains—namely the concerns of many communities in the Global South struggling with the impact of the virus in their daily lives. An example of such counter-expertise is that of the backyard poultry farmer whose ways of knowing are foreign to science and policy experts who frame the ways in which knowledge about H5N1 should be developed, controlled, and used. While the H5N1 debate illuminated competing positions regarding knowledge production between powerful elites, it ignored the social justice inequities produced by the dual-use dilemma. The concept of \emph{knowledge justice} provides a way of thinking about science that can include locally situated counter-expertise, disrupting the dual-use dilemma produced by competing dominant priorities of security and public health.


Cover image (insert image): 


Guide name: "Citizen science for all"


Guide citation insert from Zotero: \autocite{noauthor_citizen_2016}\autocite{pocock_choosing_2014}


Type of guide parts (descripe the parts, e.g. step-by-step, instructions, case study, checklists, ...)


Summary (character limit): 


\subsection{next guide to be described}\label{H7212092}



Cover image (insert image):


Guide name: "Citizen science for all"


Guide citation insert from Zotero \autocite{pocock_choosing_2014}  


\subsection{next guide to be described}\label{H9362108}



Cover image (insert image):


Guide name: Knowledge Justice: Disrupting Library and Information Studies through Critical Race Theory


Summary: 


\textbf{Schwarze, Indigene und People of Color stellen sich die Bibliotheks- und Informationswissenschaft durch die Linse der kritischen Rassentheorie neu vor.}


Die Open-Access-Ausgabe dieses Buches wurde durch die großzügige Finanzierung von Arcadia ermöglicht - einem wohltätigen Fonds von Lisbet Rausing und Peter Baldwin.


In Knowledge Justice verwenden Schwarze, indigene und People of Color Wissenschaftler die kritische Rassentheorie (CRT), um die grundlegenden Prinzipien, Werte und Annahmen der Bibliotheks- und Informationswissenschaft (LIS) in den Vereinigten Staaten in Frage zu stellen. Sie rücken CRT in den Mittelpunkt der LIS, um den Berufsstand dazu zu bringen, zu verstehen und damit zu rechnen, wie weiße Vorherrschaft Praktiken, Dienstleistungen, Lehrpläne, Räume und Richtlinien beeinflusst.


Die Autoren zeigen, dass das Fachgebiet tief in der falschen Vorstellung seiner eigenen Objektivität und Neutralität verwurzelt ist, und sie fahren fort zu zeigen, wie dies mit Annahmen über Rasse zusammenhängt. Durch tiefgreifende Analysen von Bibliotheks- und Archivsammlungen, wissenschaftlicher Kommunikation, Machthierarchien, epistemischer Vorherrschaft, Kinderbibliotheken, Lehren und Lernen, digitalen Geisteswissenschaften und dem Bildungssystem fordert Knowledge Justice die LIS heraus, sich selbst neu zu imaginieren, indem sie das Gewicht und das Erbe der weißen Vorherrschaft abwerfen und nach Rassengerechtigkeit streben.


\printbibliography[title={Literaturverzeichnis}]
\end{document}
