\documentclass{article}

\usepackage{caption}
                
\usepackage[backend=biber,hyperref=false,citestyle=authoryear,bibstyle=authoryear]{biblatex}
                
\bibliography{bibliography}
            
\usepackage{graphicx}
                
\usepackage{calc}
                
\newlength{\imgwidth}
                
\newcommand\scaledgraphics[2]{%
                
\settowidth{\imgwidth}{\includegraphics{#1}}%
                
\setlength{\imgwidth}{\minof{\imgwidth}{#2\textwidth}}%
                
\includegraphics[width=\imgwidth,height=\textheight,keepaspectratio]{#1}%
                
}
            
\begin{document}

\title{OSG005 - Open Science and Open Access Publishing }

\maketitle





\textbf{Authors}:  Maria Sael, Sabrina Gaab, Mohammad Al Nasouh, Edith Reschner


\textbf{Tags / topics} (4): Open Access, Open Science, Open Access Publishing, Open Licence, Scholarly publishing


\subsection{Template for one guide to be described}\label{H7151279}



Cover image (insert image): 





Guide name: "Open Science principles for accelerating trait-based science across the Tree of Life"


Guide citation insert from Zotero  


Type of guide parts (descripe the parts, e.g. step-by-step, instructions, case study, checklists, ...)


Summary (character limit): 


A guide written by Karen Macvean and published in the online library RMIT - Global University of Technology, Design and Business to explain everything about Open Access using various exploration methods such as text, explanatory videos, diagrams and illustrations. The guide explains the idea behind Open Access, its models, benefits, APCs, guidelines and resources. There are also tips for posting research in different types of publications such as magazines, books, or conference papers that are more "open". The guide also provides an overview of copyrights and copyrights.


\autocite{open-accessnet_platform_open_nodate} 


\subsection{next guide to be described}\label{H7212092}



Cover image (insert image):


Guide name: "Open Science principles for accelerating trait-based science across the Tree of Life"


Guide citation insert from Zotero  \autocite{gallagher_open_2020}


---------


\subsection{3. Guide}\label{H9740541}



\begin{center}
\begin{figure}
\scaledgraphics{499c4719-8346-4a67-8a5b-fcacfb0ecde0.png}{0.5}
\caption*{Abbildung 1: Cover: "Von Open Access zu Open Science"}\label{F14991031}
\end{figure}


\end{center}





\textbf{Guide name:} "Von Open Access zu Open Science: Zum Wandel digitaler Kulturen der wissenschaftlichen Kommunikation"


\textbf{Guide citation insert from Zotero:}  \autocite{heise_von_2018}


\textbf{Type of Guide: }Mit Hilfe eines Experiments werden in diesem Handbuch Chancen und Hindernisse von Open Access dargestellt. 


\textbf{Summary (Main Topics): }Mit der Digitalisierung geht der Ruf nach freiem Zugang zu wissenschaftlichen Forschungsergebnissen und einer Öffnung des Forschungsprozesses einher. Open Access und Open Science sind die Leitbegriffe dieses Transformationsprozesses, der von den einen euphorisch begrüßt und von den anderen heftig abgelehnt wird. Auf der Grundlage einer quantitativen Erhebung und eines reflexiven Experiments gibt das Buch Einblick in die aktuellen Debatten über die Chancen aber auch Hindernisse der Öffnung der Wissenschaften.





\subsection{4. Guide}\label{H1144211}


\begin{figure}
\scaledgraphics{37cc2dfd-e350-4e29-acc3-9c7f815eb133.png}{0.5}
\caption*{Cover}\label{F36070241}
\end{figure}




















\textbf{Guide name:} 

Understanding Open Access. When, why, \& how to make your work openly accessible


\textbf{Guide citation insert from Zotero:} 

\autocite{rubow_understanding_2015}


\textbf{Type of Guide:}

This Guide "Understanding Open Access" provides a scholarly author-oriented look at the ins and outs of open access publishing. The guide addresses common concerns about what "open access" means, how institutional open access requirements work, and why authors might consider making their work openly accessible online. The aim of this guide is to provide real-life strategies and tools that authors can use to work with publishers, institutions, and funders to make their works available on the terms most consistent with their dissemination goals.


This guide is the product of extensive interviews with authors, publishers, and institutional representatives who shared their perspectives on open access options in today’s publishing environment. The information, strategies, and examples included in this guide reflect the collective wisdom of these interviewees.


\textbf{Target Group:}

The guide is for authors of all backgrounds, fields, and disciplines, from the sciences to the humanities.


\textbf{Content:}

\textbf{Mabye (}Learn more about open access and related optionsComply with an open access policy from an employer or funding agency• Select the terms on which you would like to make a work openly accessible• Publish a work with an open access publisher• Make a work openly accessible on a personal or group website• Deposit a work in an open access repository• Negotiate with a conventional publisher to make a work openly accessible• And much more.) ?




\textbf{Summary (Main Topics):}














\printbibliography[title={Literaturverzeichnis}]
\end{document}
