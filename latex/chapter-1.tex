\documentclass{article}

\usepackage{hyperref}
\begin{document}

\title{Front Matter and Introduction}

\maketitle


Authors: Simon Worthington, Ina Blümel, Ludwig Hülk.


Tags / topics (4): open science, introduction, motivation


\subsection{Impressum}\label{H8810070}



Multi-format versions: \href{https://vivliostyle.vercel.app/#src=https://tibhannover.github.io/Open-Science-Guides-Collection/html/index.html&bookMode=true}{Webbook} | EPUB


Title: 


Place: Hannover, Germany


Date: April 2021


Edition / Version: Edition 1, Version \href{https://github.com/TIBHannover/Open-Science-Guides-Collection/releases/tag/v0.1-alpha}{v0.1-alpha} \#fbbb553 fbbb553f66aeb6de9e0f8c497bb503e86a65771b


DOI: 10.5281/zenodo.4672208


Open Science has become an indispensable part of modern science. There are several definitions of "openness" in relation to different aspects of science - the \href{https://opendefinition.org/}{Open Definition} sets out principles as follows “Open means anyone can freely access, use, modify, and share for any purpose (subject, at most, to requirements that preserve provenance and openness).” 


\textbf{Practical guides} for the implementation of those principles in different areas such as research data or publishing are of great importance because they can be used right away. In this compendium, we compile important guides with their specific features and fields of application. The book was written as part of a student seminar at the Hanover University of Applied Sciences and Arts.

\end{document}
