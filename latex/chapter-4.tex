\documentclass{article}

                
\usepackage{authblk}
                
\makeatletter
                
\let\@fnsymbol\@alph
                
\makeatother
            
                
\def\keywords{\vspace{.5em}
                
{\textit{Keywords}:\,\relax%
                
}}
                
\def\endkeywords{\par}
                
\newcommand{\sep}{, }
            
\usepackage{caption}
                
\usepackage[backend=biber,hyperref=false,citestyle=authoryear,bibstyle=authoryear]{biblatex}
                
\bibliography{bibliography}
            
\usepackage{graphicx}
                
\usepackage{calc}
                
\newlength{\imgwidth}
                
\newcommand\scaledgraphics[2]{%
                
\settowidth{\imgwidth}{\includegraphics{#1}}%
                
\setlength{\imgwidth}{\minof{\imgwidth}{#2\textwidth}}%
                
\includegraphics[width=\imgwidth,height=\textheight,keepaspectratio]{#1}%
                
}
            
\begin{document}

\title{Open Science and Citizen Science}

\maketitle

\author{Franziska Ahlborn}
\author{Maryna Sermus}
\affil{}


\keywords{citizien Science\sep Bildung \sep Naturschutz\sep Geisteswissenschaft\sep Community\sep Umwelt\sep UK\sep Biodiversität\sep public understanding of science\sep environment\sep biodiversity\sep nature conservation}

\subsubsection{}\label{H1731420}



\textbf{Tags / topics (4): Citizen Science; Citizen Science to monitor biodiversit}y\textbf{; Citizen Science to study biodiversity and the environment in the UK;} \textbf{Citizen Science and public understandig}


\section{Citizen science for all}\label{H2662301}



\subsection{A guide for citizen science practitioners}\label{H9328356}



\begin{center}
\begin{figure}
\scaledgraphics{3cbbd8ed-7495-46a8-8cb5-114cf95cfb83.png}{0.5}
\caption*{Citizen Science}\label{F38618731}
\end{figure}


\end{center}





\textbf{Guide citation insert from Zotero:} \autocite{noauthor_citizen_2016}


\textbf{Type: }Guide with practical instructions


\textbf{Target Audience:}


\textbf{Summary: }This guide describes how Citizen Science is practiced in Germany and how this participatory approach can be used in different research disciplines and thematic areas - such as education, nature protection or the humanities. The guide is addressed primarily to initiators of Citizen Science projects, but also to all those who participate in such projects. This includes scientists working in research institutions who want to work with citizens, but also individuals and community groups such as independent scientific groups and associations.


\textbf{Contents:} 


\textbf{Teil 1: Citizen Science Praxis}

\begin{enumerate}
\item Was ist Citizen Science?


\item Warum Citizen Science? Was sind die Vorteile? Was sind die Herausforderungen?


\item Initiierung eines Citizen Science Projekts - Auswahl von Partnern, Methoden und Teilnehmern


\item Daten: Wichtige Themen für Citizen Science Daten


\item Kommunikation und Feedback


\item Citizen Science Projekte auswerten


\item Finanzierungsinstrumente


\item Grafik: So planen Sie ein Citizen Science Projekt - von Anfang bis Ende!


\textbf{Teil 2: Citizen Science Landschaft}


\item Citizen Science im Naturschutz


\item Citizen Science und Bildung


\item Digitale Citizen Science


\item Citizen Science in den Sozialwissenschaften


\item Citizen Science in der Gesundheitsforschung


\item Citizen Science in den Kunst- und Geisteswissenschaften


\item Citizen Science International


\end{enumerate}

\section{Choosing and Using Citizen Science}\label{H1285339}



\subsection{A guide to when and how to use citizen science to monitor biodiversity and the environment}\label{H8816796}


\begin{figure}
\scaledgraphics{dba5d5eb-235c-4694-a2d4-c26b6e16182b.png}{0.5}
\label{F60064251}
\end{figure}


\textbf{Guide citation insert from Zotero:} \autocite{pocock_choosing_2014}


\textbf{Type: }Guide for a specific Citizen Science domain of application


\textbf{Target Audience:}


\textbf{Summary}: Citizen Science can serve as a very useful "tool" for conducting research and monitoring while involving many people. Citizen Science is very diverse; there are many different ways for volunteers to engage with real science.


This guide serves as a support for people considering a Citizen Science approach, particularly (but not necessarily limited to) monitoring biodiversity and the environment in the UK.


\section{Guide to Citizen Science}\label{H3514415}



\subsection{developing, implementing and evaluating citizen science to study biodiversity and the environment in the UK}\label{H903150}



Cover image (insert image):


\textbf{Guide citation insert from Zotero:} \autocite{tweddle_guide_2012}


\textbf{Type: }Guide for a specific Citizen Science domain of application


\textbf{Target Audience:}


\textbf{Summary}: Much of the UK's understanding of its flora and fauna today is based on the engagement of natural scientists. Citizen Science initiatives to collect environmental data range from crowd-sourcing activities to small groups of volunteer experts collecting and analysing environmental data and sharing their findings with others. Given the different methods of collecting data, it is important that they are well planned and executed. This will not only help science, but also promote environmental awareness among citizens.


The guide aims to support people already involved in Citizen Science and those new to it within the UK. It is based on information collected and analysed as part of the UK-EOF funded project "Understanding Citizen Science \& Environmental Monitoring".


\section{Can Citizen Science enhance the public understanding of Science?}\label{H2333653}



\begin{center}
\begin{figure}
\scaledgraphics{fb9872a4-8371-4bfe-a930-f1d621c5649e.png}{0.5}
\caption*{\textbf\{Cartoon by Tom Dunne\}}\label{F26530171}
\end{figure}


\end{center}





\textbf{Guide citation insert from Zotero:} \autocite{bonney_can_2015}


\textbf{Type: }Theoretical research work


\textbf{Target Audience:}


\textbf{Summary}: In der Publikation werden starke Belege bereitgestellt dafür, dass die wissenschaftlichen Ergebnisse von Citizen Science gut dokumentiert sind, insbesondere für Projekte zur Datenerhebung und Datenverarbeitung, auch um bei den Teilnehmern von Citizen Science-Projekten einen Wissenszuwachs über wissenschaftliche Kenntnisse und Prozesse erzielen, das öffentliche Bewusstsein für die Vielfalt der wissenschaftlichen Forschung erhöhen und den Hobbys der Teilnehmer eine tiefere Bedeutung verleihen. 


Citizen Science kann positiv zum sozialen Wohlergehen beitragen, indem es die Fragen, die behandelt werden, beeinflusst und den Menschen eine Stimme bei lokalen Umweltentscheidungen gibt. Um dies zu erreichen, erfordern Citizen Science-Projekte Anstrengungen in diesen vier Bereichen: (1) Projektdesign, (2) Ergebnismessung, (3) Einbindung neuer Zielgruppen und (4) neue Forschungsrichtungen.


\textbf{Contents:}


\subsubsection{Inhalte: }\label{H3778084}


\begin{enumerate}
\item Einleitung


\item Überblick über die Leistungen von Citizen Science

\begin{itemize}
\item Projekte zur Datenerfassung


\item Projekte zur Datenverarbeitung


\item Lehrplanbasierte Projekte


\item Gemeinschaftswissenschaft


\end{itemize}

\item Wohin geht das Feld von hier aus?

\begin{itemize}
\item Projektdesign


\item Ergebnisse Messen


\item Neue Zielgruppen ansprechen


\item Erweiterte Forschung


\end{itemize}

\item Schlussfolgerung


\end{enumerate}

\printbibliography[title={Literaturverzeichnis}]
\end{document}
