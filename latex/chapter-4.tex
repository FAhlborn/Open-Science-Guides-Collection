\documentclass{article}

                
\usepackage{authblk}
                
\makeatletter
                
\let\@fnsymbol\@alph
                
\makeatother
            
                
\def\keywords{\vspace{.5em}
                
{\textit{Keywords}:\,\relax%
                
}}
                
\def\endkeywords{\par}
                
\newcommand{\sep}{, }
            
\usepackage{caption}
                
\usepackage[backend=biber,hyperref=false,citestyle=authoryear,bibstyle=authoryear]{biblatex}
                
\bibliography{bibliography}
            
\usepackage{graphicx}
                
\usepackage{calc}
                
\newlength{\imgwidth}
                
\newcommand\scaledgraphics[2]{%
                
\settowidth{\imgwidth}{\includegraphics{#1}}%
                
\setlength{\imgwidth}{\minof{\imgwidth}{#2\textwidth}}%
                
\includegraphics[width=\imgwidth,height=\textheight,keepaspectratio]{#1}%
                
}
            
\begin{document}

\title{Open Science and Citizen Science}

\maketitle

\author{Franziska Ahlborn}
\author{Maryna Sermus}
\affil{}


\keywords{citizien Science\sep Bildung \sep Naturschutz\sep Geisteswissenschaft\sep Community\sep Umwelt\sep UK\sep Biodiversität\sep public understanding of science\sep environment\sep biodiversity\sep nature conservation}

\subsubsection{}\label{H1731420}



\textbf{Tags / topics (4): Citizen Science; Citizen Science to monitor biodiversit}y\textbf{; Citizen Science to study biodiversity and the environment in the UK;} \textbf{Citizen Science and public understandig}


\section{Citizen science for all}\label{H2662301}



\subsection{A guide for citizen science practitioners}\label{H9328356}



\begin{center}
\begin{figure}
\scaledgraphics{3cbbd8ed-7495-46a8-8cb5-114cf95cfb83.png}{0.5}
\caption*{Citizen Science}\label{F38618731}
\end{figure}


\end{center}





\textbf{Guide citation insert from Zotero:} \autocite{noauthor_citizen_2016}


\textbf{Target Group:}


\textbf{Type: }Guide with practical instructions


\textbf{Summary: }This guide describes how Citizen Science is practiced in Germany and how this participatory approach can be used in different research disciplines and thematic areas - such as education, nature protection or the humanities. The guide is addressed primarily to initiators of Citizen Science projects, but also to all those who participate in such projects. This includes scientists working in research institutions who want to work with citizens, but also individuals and community groups such as independent scientific groups and associations.


\textbf{Contents:} 


\textbf{Teil 1: Citizen Science Praxis}

\begin{enumerate}
\item Was ist Citizen Science?


\item Warum Citizen Science? Was sind die Vorteile? Was sind die Herausforderungen?


\item Initiierung eines Citizen Science Projekts - Auswahl von Partnern, Methoden und Teilnehmern


\item Daten: Wichtige Themen für Citizen Science Daten


\item Kommunikation und Feedback


\item Citizen Science Projekte auswerten


\item Finanzierungsinstrumente


\item Grafik: So planen Sie ein Citizen Science Projekt - von Anfang bis Ende!


\textbf{Teil 2: Citizen Science Landschaft}


\item Citizen Science im Naturschutz


\item Citizen Science und Bildung


\item Digitale Citizen Science


\item Citizen Science in den Sozialwissenschaften


\item Citizen Science in der Gesundheitsforschung


\item Citizen Science in den Kunst- und Geisteswissenschaften


\item Citizen Science International


\end{enumerate}

\section{Choosing and Using Citizen Science}\label{H1285339}



\subsection{A guide to when and how to use citizen science to monitor biodiversity and the environment}\label{H8816796}


\begin{figure}
\scaledgraphics{dba5d5eb-235c-4694-a2d4-c26b6e16182b.png}{0.5}
\label{F60064251}
\end{figure}


\textbf{Guide citation insert from Zotero:} \autocite{pocock_choosing_2014}


\textbf{Target Group:}


\textbf{Type: }Guide for a specific Citizen Science domain of application


\textbf{Summary}: Citizen Science can serve as a very useful "tool" for conducting research and monitoring while involving many people. Citizen Science is very diverse; there are many different ways for volunteers to engage with real science.


This guide serves as a support for people considering a Citizen Science approach, particularly (but not necessarily limited to) monitoring biodiversity and the environment in the UK.


\section{Guide to Citizen Science}\label{H3514415}



\subsection{developing, implementing and evaluating citizen science to study biodiversity and the environment in the UK}\label{H903150}



Cover image (insert image):


\textbf{Guide citation insert from Zotero:} \autocite{tweddle_guide_2012}


\textbf{Target Group: }People who have been involved in Citizen Science and people who are new to this field of science within the UK.


\textbf{Type: }Guide for a specific Citizen Science domain of application, written by scientists at the Biological Records Centre und he natural History Museum Angela Marmint Centre for UK Biodiversity, on behalf of the UK Environmental Observtion Framework.


\textbf{Parts: }The guide helps citizen who are interested in starting a project (or have already been involved in Citizen Science) step-by-step through the whole process, giving tips and examples of Citizen Science projects.


\textbf{Summary}: Much of the UK's understanding of its flora and fauna today is based on the engagement of natural scientists. Citizen Science initiatives to collect environmental data range from crowd-sourcing activities to small groups of volunteer experts collecting and analysing environmental data and sharing their findings with others. Given the different methods of collecting data, it is important that they are well planned and executed. This will not only help science, but also promote environmental awareness among citizens.


This Guide explains the different approaches to Citizen Science, the first steps to building a team, defining goals, funding the project and finding participants. It guides through the different phases of such a project: the development phase, the live phase and the phase of analysing the data, interpreting it and reporting the results.


It is based on information collected and analysed as part of the UK-EOF funded project "Understanding Citizen Science \& Environmental Monitoring".


\textbf{By Franziska Ahlborn}


\section{Can Citizen Science enhance the public understanding of Science?}\label{H2333653}



\begin{center}
\begin{figure}
\scaledgraphics{fb9872a4-8371-4bfe-a930-f1d621c5649e.png}{0.5}
\caption*{\textbf\{Cartoon by Tom Dunne\}}\label{F26530171}
\end{figure}


\end{center}





\textbf{Guide citation insert from Zotero:} \autocite{bonney_can_2015}


\textbf{Target Audience: }Researchers, those who are interested to learn the accomplishments of Citizen Science.


\textbf{Type: }Theoretical research work, written by four scientists Rick Bonney, Tina B. Phillips, Heidi L. and Jody W. Enck.


\textbf{Parts: }The research paper studies the reason why citizen science has become so widespread, explores the accomplishments of Citizen Science its the different categories and the four categories in which effort and resources are needed for projects to expand their influence.


\textbf{Summary}: The publication provides strong evidence that the scientific outcomes of Citizen Science are well documented, especially for data collection and processing projects. Furthermore Citizen Science achieves knowledge growth about scientific knowledge and processes among its participants, increases public awareness on the diversity of scientific research, and gives deeper meaning to participants' hobbies.


Citizen Science can contribute positively to social well-being by influencing the issues being addressed and giving people a voice in local environmental decisions. To achieve this, Citizen Science projects require efforts in these four areas: (1) project design, (2) outcome measurement, (3) engaging new audiences, and (4) new research directions.


\textbf{By Franziska Ahlborn}


\printbibliography[title={Literaturverzeichnis}]
\end{document}
