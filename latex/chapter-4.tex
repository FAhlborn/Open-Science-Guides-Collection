\documentclass{article}

\usepackage{caption}
                
\usepackage[backend=biber,hyperref=false,citestyle=authoryear,bibstyle=authoryear]{biblatex}
                
\bibliography{bibliography}
            
\usepackage{graphicx}
                
\usepackage{calc}
                
\newlength{\imgwidth}
                
\newcommand\scaledgraphics[2]{%
                
\settowidth{\imgwidth}{\includegraphics{#1}}%
                
\setlength{\imgwidth}{\minof{\imgwidth}{#2\textwidth}}%
                
\includegraphics[width=\imgwidth,height=\textheight,keepaspectratio]{#1}%
                
}
            
\begin{document}

\title{OSG004 - Open Science and Citizen Science}

\maketitle


Authors: 


Tags / topics (4): citizen science; 


\subsection{Template for one guide to be described}\label{H7151279}



Cover image (insert image): 


\begin{center}
\begin{figure}
\scaledgraphics{2decc25d-9faa-4f12-8b5b-f43d663b02d7.jpg}{0.5}
\caption*{Abbildung 1}\label{F41123511}
\end{figure}


\end{center}


Guide name: "Citizen science for all"


Guide citation insert from Zotero: \autocite{noauthor_citizen_2016}\autocite{pocock_choosing_2014}


Type of guide parts (descripe the parts, e.g. step-by-step, instructions, case study, checklists, ...)


Summary (character limit): This is a guide I found on the web. \autocite{noauthor_citizen_2016}


\subsection{next guide to be described}\label{H7212092}



Cover image (insert image):


Guide name: "Citizen science for all"


Guide citation insert from Zotero \autocite{pocock_choosing_2014}  


\printbibliography[title={Literaturverzeichnis}]
\end{document}
